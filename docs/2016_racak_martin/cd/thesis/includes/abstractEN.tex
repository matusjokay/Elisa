Creation of school timetables is a complex problem which affects most
educational institutions including \acrshort{fei} \acrshort{stu}. This is one of
the reasons why master thesis Timetable Creator for Universities (Knapereková,
2014) contains a prototype of timetabling system which aspires to solve this
problem and serves as the basis for our thesis. We familiarized ourselves with
issues connected to timetabling and requirements for timetable creation at
\acrshort{fei}. We updated the prototype and devised an approach to its
following development. The system was split into an independent server and
client which brings with itself several benefits and increases flexibility of
the whole system. The result of implementing the proposed changes is prototype
of a web service written in Python and Django framework and a web client written
in Elm that communicate together through a \acrshort{rest} \acrshort{api}.
