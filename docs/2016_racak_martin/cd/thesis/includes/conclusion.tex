Autorka práce Rozvrhový systém pre vysoké školy navrhla a implementovala
prototyp rozvrhového systému, ktorý by mal v konečnom dôsledku úspešne
konkurovať iným systémom tohto typu. \cite{knap} Cieľom našej práce bolo
aktualizovať v ňom použité technológie a pokračovať v jeho vývoji.

Najskôr sme sa informovali o probléme tvorby školských rozvrhov a niekoľkých
jeho riešeniach v podobe už existujúcich systémov. Oboznámili sme sa so súčasným
procesom tvorby rozvrhov na \acrshort{fei} a požiadavkami, ktoré sa na tieto
rozvrhy kladú. Naštudovali sme si prácu Rozvrhový systém pre vysoké školy
vrátane zdrojových kódov prototypu systému, ktorý v rámci nej vznikol a
vyskúšali si jeho používanie.

Okrem aktualizácie v ňom použitých technológií sme navrhli niekoľko možností ako
tento systém zlepšiť. Rozhodli sme sa pozmeniť architektúru systému a
pretransformovať ho na webovú službu s \acrshort{rest} \acrshort{api} a webového
klienta vo forme \acrshort{spa} s cieľom zvýšiť jeho interoperabilitu s inými
systémami a umožniť vytvorenie viacerých klientov napr. vo forme natívnych
(mobilných) aplikácií. Dôsledkom toho je ešte o čosi vyššia modularita celého
systému, ktorá už pri jeho pôvodnom návrhu zohrávala dôležitú rolu. Takéto
oddelenie tiež uľahčuje interaktívnu manipuláciu s dátami na strane klienta a
znižuje odozvu používateľského rozhrania. Dátový model na strane servera sme z
väčšej časti, ponechali pôvodný. Popísali sme technológie, ktoré sme použili na
realizáciu nášho návrhu a odôvodnili ich výber. Priblížili sme si postup pri
aktualizácii a ďalšom vývoji prototypu. Detailnejšie sme opísali zaujímavé
aspekty implementácie systému a jeho fungovania.

Softvér zatiaľ stále nie je vhodný na nasadenie, čo je spôsobené jednak celkovou
komplexnosťou problému a tiež tým, že s veľkou časťou použitých technológií sme
sa stretli prvý krát. Existuje značný priestor na jeho ďalšie vylepšenie a jeho
kompletizácia a prípadné nasadenie na \acrshort{fei} si zrejme bude vyžadovať
dlhodobé úsilie a úzku spoluprácu s personálom fakulty. Zmena architektúry
systému však otvára mnohé nové možnosti pre jeho ďalší vývoj, napr.:

\begin{itemize}
\item vývoj servera a klienta môže prebiehať oddelene,
\item vývoj viacerých nezávislých klientov pre rôzne platformy,
\item klient môže poskytovať offline funkcionalitu,
\item integrácia servera s inými službami, napr. \acrshort{ais} alebo službami
  tretích strán.
\end{itemize}
