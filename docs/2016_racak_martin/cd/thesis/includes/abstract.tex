Tvorba školských rozvrhov je komplexný problém, ktorý trápi väčšinu vzdelávacích
inštitúcií vrátane \acrshort{fei} \acrshort{stu}. Aj z tohto dôvodu vznikol v
rámci diplomovej práce Rozvrhový systém pre vysoké školy (Knapereková, 2014)
prototyp rozvrhového systému, ktorý má za cieľ tento problém riešiť a z ktorého
sme v našej práci vychádzali. Oboznámili sme sa s problematikou rozvrhovania a
požiadavkami kladenými na tvorbu rozvrhov na \acrshort{fei}. Prototyp systému
sme aktualizovali a navrhli sme postup jeho ďalšieho vývoja. Systém sme
rozdelili na samostatnú serverovú a klientskú časť, čo zo sebou prináša viacero
výhod a zvyšuje flexibilitu celého systému. Výsledkom implementácie navrhnutých
zmien je prototyp webovej služby, napísanej v Pythone a frameworku Django a
webového klienta, napísanom v Elme, ktoré spolu komunikujú cez \acrshort{rest}
\acrshort{api}.
