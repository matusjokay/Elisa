Rozvrhový systém je softvér, ktorý slúži na zhotovenie rozvrhu. Vo vzdelávacích
inštitúciách sa rozvrh používa na alokáciu študentov, učiteľov a miestností
(priestorov) do časovo ohraničených udalostí. Vytvoriť rozvrh tak, aby sme
vylúčili, alebo aspoň minimalizovali, kolízie medzi jeho jednotlivými prvkami a
zároveň dodržali požadované obmedzenia je spletitý problém. Veľké požiadavky na
takýto systém majú obzvlášť vysoké školy a univerzity, kde sa rozvrh tvorí
spravidla niekoľko krát za rok, na semester aj skúškové obdobie, pri pomerne
veľkom počte vstupných parametrov.

V našej práci sme sa oboznámili s viacerými existujúcimi rozvrhovými systémami,
vrátane prototypu, ktorý vznikol ako súčasť diplomovej práce Rozvrhový systém
pre vysoké školy (Emília Knapereková, 2014) \cite{knap}. Navrhli sme niekoľko
možností ako ho vylepšiť a pokračovali sme v jeho vývoji s dôrazom na potreby
\acrshort{fei} \acrshort{stu}, kde by mal byť aj nasadený, s cieľom zvýšiť
efektivitu procesu tvorby rozvrhu na fakulte.

V kapitole \ref{sec:analysis} analyzujeme problém tvorby rozvrhov a niekoľko
softvérových systémov, ktoré ho riešia. Kapitola \ref{sec:goal} obsahuje
sumarizáciu požiadaviek na rozvrhový systém a opis súčasného stavu rozvrhovania
na \acrshort{fei}. Náš návrh ďalšieho postupu pri aktualizácii a vývoji
prototypu systému sme opísali v kapitole \ref{sec:design}. Nasleduje prehľad
použitých technológií v kapitole \ref{sec:used_technologies}. Postup pri
implementácii zmien a fungovanie systému sme popísali v
kapitole~\ref{sec:implementation}.
