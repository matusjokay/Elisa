Na vývoj systému by malo vyhovovať akékoľvek prostredie, ktoré podporuje Python
3, PostgreSQL a Node.js. Pre produkčné prostredie postačuje Python 3 a
PostgreSQL. Postup ako nastaviť vývojové prostredie sme popísali v súboroch
\texttt{README.md}, ktoré sa nachádzajú v priečinku so zdrojovými súbormi
servera, resp. klienta na priloženom nosiči. Prostredie, ktoré sme používali na
vývoj my, je popísane v podkapitole \ref{subsec:develop_env}.

Webovú službu aj klienta je možné spúšťať lokálne s použitím vývojových webových
serverov, ktoré sú súčasťou ich závislostí. Ale na nasadenie systému je potrebné
použiť plnohodnotný webový server ako napr. Apache HTTP Server alebo Nginx. Kód
klienta stačí zostaviť vo vývojovom prostredí a na server iba nahrať výsledné
statické súbory (\acrshort{html}, \acrshort{css}, \acrshort{js} atď.), ktoré
môže servírovať ten istý server ako webovú službu. Nasadenie služby odporúčame
konzultovať s Django
dokumentáciou\footnote{\url{https://docs.djangoproject.com/en/1.11/howto/deployment/}}.
Je potrebné zabezpečiť, aby všetka komunikácia so serverom prebiehala cez
\acrshort{https}, na čo navrhujeme využiť Let's
Encrypt\footnote{\url{https://letsencrypt.org/}}.
