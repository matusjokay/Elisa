Hlavným cieľom našej práce bolo aktualizovať existujúci stav projektu a pokračovať
v jeho vývoji, s dôrazom kladeným na používateľské rozhranie.

Najskôr sme analyzovali problematiku tvorby rozvrhov a oboznámili sme sa s niektorými
existujúcimi riešeniami. Zosumarizovali sme si celý proces tvorenia rozvrhu na FEI 
a jednotlivé požiadavky kladené na systém. Z oboch predchádzajúcich prác sme si osvojili aktuálny stav riešenia
a navrhli zmeny potrebné na docielenie stanovených cieľov používateľského rozhrania systému.

Našou navrhovanou zmenou boli používané technológie a zmena architektúry v klientskej časti.
Pripomenuli sme si používané nemenené technológie na strane servera z predchádzajúcej diplomovej
práce a popísali sme si jednotlivé nami navrhované technológie.
Po rozbehaní systému z predchádzajúcej práce sme sa pustili do implementácie nového používateľského rozhrania.
Priblížili sme si implementáciu komponentov, z ktorých sme vyskladávali jednotlivé moduly a taktiež implementáciu
niektorých kľučových funkcionalít.

Systém je po oboch predchádzajúcich prácach a teraz po pridaní našej časti na veľmi dobrej ceste
k používaniu v budúcnosti, no vyžaduje si ešte veľmi veľa práce a úsilia.
Softvér stále nie je plne dokončený, nakoľko sa jedná o skutočne robustný systém s množstvom
potrebných a plánovaných funkcionalít. Samotný vývoj si vyžadoval veľmi veľa času, nakoľko sme sa rozhodli
pre súčasné moderné technológie a momentálne je ešte pomerne náročné nájsť k nim relevantné zdroje
a podrobné postupy v komunite na internete. Taktiež splnenie požiadaviek na responzívnosť nám zabralo približne 30\%
času celého vývoja. To sa nám však podarilo a do budúcna nevznikne potreba pre vznik mobilnej aplikácie systému.

Ako návrh na ďalší postup by sme uviedli potrebu zapracovať na tzv. middleware vrstve, a to na prepojení 
serverovej časti implementovanej v predchádzajúcej práci s klienstkou časťou implementovanou v našej práci.

Medzi možné vylepšenia systému navrhujeme:
\begin{itemize}
\item Offlina funkcionalita
\item Zapracovanie používateľských práv do množiny ponúkaných funkcionalít
\item Integrácia systému so systémom AIS (napr. v komunikačnom moduli)
\item Prepojenie doplnkových informačných komponentov na služby tretích strán
\end{itemize}