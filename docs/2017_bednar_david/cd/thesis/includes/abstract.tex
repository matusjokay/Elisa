Tvorenie rozvrhov je rozsiahly a časovo náročný proces, ktorým sa každoročne zaoberá väčšina vzdelávacích inštitúcií vrátane FEI. Kvôli tejto motivácii vznikol v roku 2014 v rámci diplomovej práce  (Emília Knapereková, 2014) prototyp rozvrhového systému pre vysoké školy. Na základe tohto prototypu sa neskôr v ďalšej diplomovej práci vytvorila základná aplikácia rozvrhu (Martin Račák, 2017). Cieľom tejto práce je aktualizácia projektu a pokračovanie v jeho vývoji, s dôrazom kladeným na používateľské rozhranie systému. Oboznámili sme sa s predošlými prácami na tomto systéme a stav implementácie rozhrania sme zmenili z funkcionálnej v jazyku Elm na objektovú v jazyku Angular 5 s obohatením o prvky funkcionálneho programovania pomocou Redux manažmentu stavov. Hlavným prínosom systému má byť jednoduchosť a intuitívnosť pri jeho používaní, a teda aj jednotlivé funkcie a možnosti sú orientované na tieto požiadavky.
V závere našej práce sa nachádza zhodnotenie jej prínosu a dosiahnuté výsledky v porovnaní so stanovenými cieľmi na začiatku projektu.