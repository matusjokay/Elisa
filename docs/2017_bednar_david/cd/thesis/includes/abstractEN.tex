Schedule creation is an extensive and time consuming process that is being undertaken annually by most educational institutions including FEI. Due to this motivation, first prototype of a university schedule system (Emília Knapereková, 2014) was created in the master thesis in 2014. Based on this prototype, the basic application of the schedule was created later in the next master thesis (Martin Račák, 2017). We have become familiar with previous work on this system, and the state of implementation of the interface has changed from functional in Elm to object oriented in Angular 5 with enhancement of Functional Programming elements using Redux State Management. The main benefits of the system are the simplicity and intuitiveness of its use, and therefore the various functions and options are oriented to these requirements. At the end of our work, we evaluated its benefits and achieved results compared to the goals set at the beginning of the project.