\enlargethispage{1cm}
V minulosti, pred etablovaním počítačov do života ľudí, sa väčšina činností,
ako aj zostavovanie rozvrhov, vykonávala manuálne. Rozvrhár musel ručne
spracovávať zhromaždené údaje, odsledovať kolízie a nastaviť celý proces tak,
aby správne fungoval a dalo sa podľa rozvrhu spoľahlivo vyučovať. Pri vysokých
školách je tento problém potrebné riešiť aj viac ako dvakrát do roka kvôli rôznorodosti
výučbového obdobia (zimný / letný semester, semester výučby / skúškové obdobie).
Komplexnosť (počet vyučujúcich, študentov, predmetov a miestností), jednotlivé
pravidlá (všeobecné aj osobitné) a rozmanitosť výnimiek celý proces ešte viac komplikujú.
V tejto sfére sa počítače stali vhodným nástrojom pri zozbieraní dát, riešení problémov a
kolízií, generovaní rozvrhov a za pomoci internetu aj v jednoduchom distribuovaní online
pre koncových používateľov, ako sú samotní vyučujúci či študenti. Tak, ako sa časom
vyvíjali takéto systémy, sa vyvíjali aj nové štandardy, technológie a trendy v oblasti
vývoja informačných systémov.
Existuje veľký počet komerčných softvérových produktov, ktoré pomáhajú
s problémom tvorenia rozvrhov, avšak kvôli rôznorodosti, špecifickým požiadavkám,
postupom a rôznej zložitosti každej inštitúcie sa stále vedú zaujímavé štúdie v oblasti
plánovania procesov pri tvorbe rozvrhov ako aj v oblasti dizajnu a návrhu rozhrania
pre koncových používateľov.

\vskip 0.5cm
V prvej kapitole našej práce sa venujeme analýze problému. Popisujeme všeobecný proces
tvorby rozvrhov a zameriavame sa na vystupujúce entity a fakty, ktoré sú zohľadňované
pri výstupoch výsledných rozvrhov. Analyzujeme stav existujúcich riešení, ktoré sa 
v súčasnosti využívajú na rozvrhovanie s dôrazom na ich funkcionalitu, možnosti a náhľad
používateľského rozhrania.
V druhej kapitole sme sa zamerali na zosumarizovanie cieľov práce na základe špecifikácie
požiadaviek kladených na systém. Podrobnejšie popisujeme spôsob, ako v súčasnosti
prebieha proces tvorby rozvrhu na našej fakulte FEI.
Tretiu časť diplomovej práce venujeme samotnému návrhu riešenia. Od popisu vysokoúrovňovej
architektúry sa jednotlivými podkapitolami približujeme ku konkrétnym architektonickým prvkom
používaných v aplikácii.
V štvrtej kapitole predstavujeme použité technológie, pre ktoré sme sa rozhodli pri implementácii
navrhovaného systému.
Poslednú piatu kapitolu venujeme samotnej implementácii riešenia a popisu jednotlivých modulov
spolu s ukážkami a vysvetleniami základných najdôležitejších prvkov aplikácie.

\vskip 0.5cm
V práci budeme na označenie cudzojazyčných výrazov, prípadne na označenie názvov modulov,
ktoré je kvôli porozumeniu lepšie neprekladať, používať písmo štýlu \textit{italic}. Príklady príkazov
budeme pre zvýraznenie od bežného textu odlišovať písmom \texttt{technického štýlu}.



