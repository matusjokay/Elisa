Systém si na svoju funkčnosť vyžaduje prostredie, na ktorom je podporovaný Python 3
a databáza PostgreSQL. V prípade potreby vyvíjať bude nevyhnutná inštalácia Node.js.
Postup konfigurácie a spustenia vývojového prostredia sa nachádza popísaný na priloženom
médiu v priečinku so zdrojovými súbormi aplikácie v súbore README.md jednotlivo pre server
a pre klienta. Pre prípad potreby sme nami použivané prostredie popísali v kapitole
\ref{subsec:develop_env}.

Na vývoj, prípadne testovanie, je možné aplikáciu spúšťať lokálne prostredníctvom web serverov.
Na samotné produkčné nasadenie celého systému je však potrebné použitie reálneho plnohodnotného
servra, na ktorý sa následne nahrajú všetky potrebné statické súbory výslednej aplikácie
(HTML, CSS, JS, ...) a budú servírované ako webové služby. Pri tomto nasadení odporúčame
postupovať podľa platnej dokumentácie Django
\footnote{\url{https://docs.djangoproject.com/en/1.11/howto/deployment/}}.

Kvôli bezpečnosti systému je vhodné na komunikáciu medzi klientom a servrom využiť niektorú
so šifrovacích služieb, napr. službu Let's Encrypt\footnote{\url{https://letsencrypt.org/}}.

